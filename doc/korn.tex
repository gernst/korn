\documentclass{llncs}
\usepackage[utf8]{inputenc}
\usepackage[T1]{fontenc}
\usepackage[scaled=0.8]{beramono}
\usepackage{amsmath}
\usepackage{amssymb}
\usepackage{stmaryrd}
\usepackage{mathtools}
\usepackage{tabto}
\usepackage{comment}
\usepackage{xspace}

\usepackage{hyperref}
\hypersetup{allcolors=blue,colorlinks=true}

\usepackage{cleveref}
% \usepackage[numbers,sort&compress]{natbib}
\newcommand{\mailto}[1]{\href{mailto:#1}{\ttfamily #1}}

\newcommand{\Korn}{\textsc{Korn}\xspace}
\newcommand{\err}{\lightning}
\newcommand{\brk}[1]{{#1}_\downarrow}
\newcommand{\False}{\mathit{false}}
\newcommand{\True}{\mathit{true}}

\author{Gidon Ernst\thanks{Jury Member}}
\title{\Korn---Software Verification \\ with Invariants and Summaries}
\institute{LMU Munich \\ \mailto{gidon.ernst@lmu.de}}
\pagestyle{plain}

\begin{document}

\maketitle

\begin{abstract}
    \Korn is a software verifier that infers correctness certificates automatically using state-of-the-art Horn-clause solvers, such as Z3 and Eldarica.
    The novel aspect is that it uses not only invariants but also summaries, which are a fundamental and principled technique, complementary to invariants~\cite{hehner1999refinement,hehner2005specified,tuerk2010local}.
    More details about the approach can be found in~\cite{ernst:arxiv2020:summaries}.
    The tool is available at \url{https://github.com/gernst/korn}

\end{abstract}
    \keywords{Software Verification \and Horn clauses \and Loop Specifications}

\section{Verification Approach}
\label{sec:approach}

\Korn is a verifier for C programs that is based on a translation into systems of Horn clauses
\cite{bjorner2015horn,gurfinkel2019science}, similarly to SeaHorn~\cite{gurfinkel2015seahorn}.
Such Horn clause systems have unknown predicates that characterize sets of states,
or relations between states, together with additional constraints that encode the correctness of the program.

Traditionally, loops are abstracted by invariants, which need to hold in the states
encounteded throughout the execution of a loop, and which characterize the work done by the loop so far.
\Korn additionally supports a second verification approach (not used in SV-COMP 2021, cf \cref{sec:project}), relying on loop summaries that instead characterize
the work that remains to be done until completion of the loop~\cite{hehner1999refinement,tuerk2010local,hehner2005specified}
(also called loop postconditions).
As shown in~\cite{ernst:arxiv2020:summaries}, summaries are a fundamental and principled technique dual to invariants,
which is relatively complete under certain conditions, analogously to~\cite{hoare1969axiomatic}.

\Korn uses state-of-the-art solvers to determine the satisfiability of the generated Horn clause system
using off-the-shelf solvers, specifically here Z3~\cite{bjorner2013solving} and Eldarica~\cite{hojjat2018eldarica},
the top performers in the latest CHC competition~\cite{rummer2020competition},
Both solvers generate evidence for correctness of a given program in terms of models that describe how the unknown predicates need to be instantiated.
Moreover, Eldarica can generate counterexample traces, and \Korn instruments the Horn clause system to get
the concrete values returned by the \texttt{\_\_VERIFIER\_nondet\_*()} functions on an error path.

The encoding of C programs in \Korn is currently subject to several limitations,
which can cause unsound and/or imprecise analysis in some corner cases.
The different solvers have different strengths and weaknesses, too,
and may fail to find proofs, claim false counterexamples (e.g. when no \emph{non-recursive} model exists),
and sometimes fail to find trivial counterexamples.
To that end, \Korn implements a portfolio approach with several sequential stages.
\begin{enumerate}
    \item Initially, a short round of random sampling with small values is performed.
          The concrete scheme picks for each nondeterministic choice
          uniformly among four options: constant 0 or a random value of either 1, 5, or 10 bits.
          This approach prevents long running loops and overall turned out to be enormeously effective
          (> 100 solved benchmarks within a few seconds each).
          Random sampling can sometimes bypass unsound verdicts,
          which happens in a single instance in the competition (cf. \cref{sec:discussion}).
    \item Next, Z3 is executed on the verification problem, translated from C to Horn clauses,
          again for a short time. Empirically, Z3 finds most solutions very quickly if it succeeds,
          including those where Z3 succeeds but not Eldarica.
    \item Finally, Eldarica is executed for the remaining time.
          It is slightly better in comparison to Z3 in the long run on this specific set of tasks~\cite{ernst:arxiv2020:summaries}.
    More importantly, the generated invariants from Eldarica tend to be simpler and avoid the existential quantifiers
    often introduced by Z3, which improves witness generation.
    To prevent spurious counterexamples,
    \Korn reports a violation of the specification only if it can be confirmed by executing the program natively.
\end{enumerate}


\section{Software Architecture}
\label{sec:architecture}

\Korn is written in the JVM language Scala%
    \footnote{\url{https://scala-lang.org}}, with the exception of the random sampler that is written in C.
The translation from C to Horn clauses supports a substantial fraction of the C~language.
Verification conditions are generated in the fragment of SMT-LIB of the \texttt{HORN} logic.%
    \footnote{\url{https://chc-comp.github.io/format.html}}
It can invoke any compliant solver as a backend either using its standard input or a file to communicate the verification task.
\Korn comes with some explicit support for Z3~\cite{gurfinkel2019science} and Eldarica~\cite{hojjat2018eldarica}
to pass e.g. timeouts or specific options to produce models resp. counterexamples.
Currently, \Korn makes use of the theory of integers (possibly using modulo and division of the C program uses it) and the theory of arrays.

Different translation schemes for blocks, invariants, summaries, ...

Easy to hack :)

In order to produce SV-COMP correctness witnesses, \Korn can read the models generated by the backend-solvers, and translate them back into C expressions.
The correctness witnesses produced currently are derived entirely from invariants (summaries would require quantifiers).

Violation witnesses 

random sampling implementation

\section{Discussion of Strengths and Weaknesses of the Approach}
\label{sec:discussion}

, including msot of the control structures including labels and goto.
Arrays are treated as value types without bounds, which is unsound in general but sufficient as a work-around for the categories in which \Korn participates.
There is no model of the heap currently, and thus there is no support for dynamic memory allocation

Evaluation of the results for the benchmark categories, where was the checker successful, where not, why?

Random sampling cool! PRevents spurious proofs


\section{Tool Setup and Configuration}
\label{sec:project}

The implementation of \Korn is available at
    \url{https://github.com/gernst/korn} under the MIT license,
the version running in SV-COMP 2020 was packaged from commit
\href{https://github.com/gernst/korn/commit/767eca718edf7f04e995142b679762242c68eef5}{\tt 767eca7}
and shows version number \texttt{0.3}.

\Korn can be downloaded and installed from the repository, type \texttt{make} to compile,
more detailed instructions can be found in the \texttt{README.md}.
All dependencies, except for Java~11 and \texttt{gcc}, are included or will be downloaded automatically.
After compilation, you can run \texttt{./korn.sh} or \texttt{./run}
(where the latter includes the current directory \texttt{.} in \texttt{PATH}).
A pre-built, ready-to-run
\href{https://gitlab.com/gernst/svcomp-archives-2021/-/blob/master/2021/korn.zip}{archive}
is with the competition version is available in the official repository.

\paragraph{Participation.} \Korn participates in four categories
\texttt{ControlFlow},
\texttt{Loops},
\texttt{Recursive},
\texttt{XCSP}
for property \texttt{ReachSafety}.

\paragraph{Configuration.}
Originally, it was planned to run \Korn using the summary-based translation in SV-COMP 2021,
but it is unclear how to generate correctness witnesses for these.
A constructive translation from summaries to invariants exists~\cite[Prop. 1]{ernst:arxiv2020:summaries},
however this comes at the expense of additional quantifiers which cannot be faithfully represented in witnesses.
Hence, a standard translation using invariants only was chosen.

\Korn ran with the following options
(which are spelled out in their long form in the benchmark \texttt{xml} file).

\smallskip

\texttt{-i} (implicit)
    \tabto{2.8cm} generate verification conditions with invariants only

\texttt{-w}
    \tabto{2.8cm} write SMT-LIB file to disk (required for Eldarica)

\smallskip

\texttt{-model -witness witness.graphml}
    produce model, write SV-Comp witness

\smallskip

\texttt{-r~~10}
    \tabto{2.8cm} Try to find obvious bugs by random sampling for at most 10s

\texttt{-t~~20 -z3}
    \tabto{2.8cm} Try Z3 4.8.9 subsequently for at most 20s

\texttt{-t~900 -eld}
    \tabto{2.8cm} Run Eldarica 2.0.4 for the remaining time

\smallskip

\noindent
The arguments to the backend solver are shown in the command line output.

\paragraph{Contributors \& Acknowledgement.}

\Korn is developed and maintained by the author.
The source code is available via \url{https://github.com/gernst/korn} under the MIT license.
Gregor Alexandru~\cite{alexandru2019} and Johannes Blau have contributed
key insights to the methodology outlined in \cref{sec:approach}.

\bibliographystyle{splncs04}
\bibliography{korn.bib}

\end{document}
