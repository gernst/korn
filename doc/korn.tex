\documentclass{llncs}
\usepackage[utf8]{inputenc}
\usepackage[T1]{fontenc}
\usepackage[scaled=0.8]{beramono}
\usepackage{amsmath}
\usepackage{amssymb}
\usepackage{stmaryrd}
\usepackage{mathtools}
\usepackage{tabto}
\usepackage{comment}
\usepackage{xspace}

\usepackage{hyperref}
\hypersetup{allcolors=blue,colorlinks=true}

\usepackage{cleveref}
% \usepackage[numbers,sort&compress]{natbib}
\newcommand{\mailto}[1]{\href{mailto:#1}{\ttfamily #1}}

\newcommand{\Korn}{\textsc{Korn}\xspace}
\newcommand{\err}{\lightning}
\newcommand{\brk}[1]{{#1}_\downarrow}
\newcommand{\False}{\mathit{false}}
\newcommand{\True}{\mathit{true}}

\author{Gidon Ernst\thanks{Jury Member}}
\title{\Korn---Software Verification \\ with Invariants and Summaries}
\institute{LMU Munich \\ \mailto{gidon.ernst@lmu.de}}

\begin{document}

\maketitle

\begin{abstract}
    \Korn is a software verifier that infers correctness certificates automatically using state-of-the-art Horn-clause solvers, such as Z3 and Eldarica.
    The novel aspect is that it uses not only invariants but also summaries, which are a fundamental and principled technique, complementary to invariants~\cite{hehner1999refinement,hehner2005specified,tuerk2010local}.
    More details about the approach can be found in~\cite{ernst:arxiv2020:summaries}.
    The tool is available at \url{https://github.com/gernst/korn}

\end{abstract}
    \keywords{Software Verification \and Horn clauses \and Loop Specifications}

\begin{comment}
\section{Verification Approach}
\label{sec:approach}
1
\begin{align*}
    P(s_0)
        & \implies I(s_0,s_0) \\
    I(s_0,s) \land t(s) \land B(s,\err)
        & \implies \False \\
    I(s_0,s) \land t(s) \land B(s,\brk s')
        & \implies Q(s') \\
    I(s_0,s) \land t(s) \land B(s,s')
        & \implies I(s_0,s') \\
    I(s_0,s) \land \lnot t(s)
        & \implies Q(s)
\end{align*}

\cite{hoare1969axiomatic}
A short overview of the theory that the tool is based on. Description of the abstract domains and algorithms that are used. Reference to the concept papers that describe the technical details.

\section{Software Architecture}
\label{sec:architecture}

\Korn is written in Scala and runs on the JVM,
the frontend is hand-crafted with JFlex and Beaver,
and supports a substantial fraction of the C~language.

Verification conditions are generated in the fragment of SMT-LIB of the \texttt{HORN} logic.
It can invoke any compliant solver as a backend either using its standard input or a file to communicate the verification task.
\Korn comes with some explicit support for Z3~\cite{gurfinkel2019science} and Eldarica~\cite{hojjat2018eldarica}
to pass e.g. timeouts or specific options to output models.

Currently, \Korn makes use of the theory of integers (possibly using modulo and division of the C program uses it),
and also the theory of arrays (C arrays are treated as value types so far, which is unsound in the presence of aliasing).

\section{Discussion of Strengths and Weaknesses of the Approach}
\label{sec:discussion}

Evaluation of the results for the benchmark categories, where was the checker successful, where not, why?

\section{Tool Setup and Configuration}
\label{sec:project}

The implementation of \Korn is available at
    \url{https://github.com/gernst/korn} under the MIT license.

    Download instructions (a public web page from which the tool can be downloaded) including a reference to a precise version of the tool (do not refer to ``the latest version'' or such, because that is not stable and not replicable)
    Installation instructions

\paragraph{Participation.} \Korn participates in the category \texttt{ReachSafety-loops}.

\paragraph{Configuration.} \Korn is run with the following options

\medskip

\texttt{-s}
    \tabto{2cm} generate verification conditions that include summaries

\medskip

\texttt{-q}
    \tabto{2cm} reduce output to one of \\
    \tabto{2cm} \texttt{sat}   \tabto{3cm} (program is safe) \\
    \tabto{2cm} \texttt{unsat} \tabto{3cm} (error can be reached) \\
    \tabto{2cm} \texttt{error} \tabto{3cm} (task is currently unsupported)

\medskip

\texttt{-eld}
    \tabto{2cm} use the Eldarica~\cite{hojjat2018eldarica} solver, packaged as version 2.0.4, invoked with \\
    \tabto{2cm} \texttt{java -Xmx14g -Xss20m -cp eld.jar lazabs.Main <file>}
    

    
\paragraph{Contributors \& Acknowledgement.}

\Korn is developed and maintained by the author.
The source code is available via \url{https://github.com/gernst/korn} under the MIT license.

Gregor Alexandru~\cite{alexandru2019} and Johannes Blau have contributed
key insights to the methodology outlined in \cref{sec:approach}.
\end{comment}

\bibliographystyle{splncs04}
\bibliography{korn.bib}

\end{document}
