\documentclass{llncs}
\usepackage[utf8]{inputenc}
\usepackage[T1]{fontenc}
\usepackage[scaled=0.8]{beramono}

\usepackage{tabto}
\usepackage{xspace}

\usepackage{hyperref}
\hypersetup{allcolors=blue,colorlinks=true}

\usepackage{cleveref}
% \usepackage[numbers,sort&compress]{natbib}
\newcommand{\mailto}[1]{\href{mailto:#1}{\ttfamily #1}}

\newcommand{\Korn}{\textsc{Korn}\xspace}

\author{Gidon Ernst}
\title{\Korn - Software Verification \\ with Invariants and Summaries}
\institute{LMU Munich \\ \mailto{gidon.ernst@lmu.de}}

\begin{document}

\maketitle

\begin{abstract}
    \Korn is a software verification tool that translates
    C~programs into systems of Horn clauses
    and uses off-the-shelf backends to solve these.
    In contrast to prior approaches, loops are encoded using
    not just invariants but also summaries.
    Invariants characterize a partial result computed by a loop so far.
    Summaries in contrast capture the complete computation in relation to some intermediate state.
\end{abstract}

\section{Verification Approach}
\label{sec:approach}

\cite{hoare1969axiomatic}
A short overview of the theory that the tool is based on. Description of the abstract domains and algorithms that are used. Reference to the concept papers that describe the technical details.

\section{Software Architecture}
\label{sec:architecture}

    Libraries and external tools that the verification tool uses (e.g., parser frontend, SAT solver)
    Software structure and architecture (e.g., components that are used in the competition)
    Implementation technology (e.g., programming language)

\section{Discussion of Strengths and Weaknesses of the Approach}
\label{sec:discussion}

Evaluation of the results for the benchmark categories, where was the checker successful, where not, why?

\section{Tool Setup and Configuration}
\label{sec:project}

The implementation of \Korn is available at
    \url{https://github.com/gernst/korn} under the MIT license.

    Download instructions (a public web page from which the tool can be downloaded) including a reference to a precise version of the tool (do not refer to ``the latest version'' or such, because that is not stable and not replicable)
    Installation instructions

\paragraph{Participation.} \Korn participates in the category \texttt{ReachSafety-loops}.

\paragraph{Configuration.} \Korn is run with the following options

\medskip

\texttt{-s}
    \tabto{2cm} generate verification conditions that include summaries

\medskip

\texttt{-q}
    \tabto{2cm} reduce output to one of \\
    \tabto{2cm} \texttt{sat}   \tabto{3cm} (program is safe) \\
    \tabto{2cm} \texttt{unsat} \tabto{3cm} (error can be reached) \\
    \tabto{2cm} \texttt{error} \tabto{3cm} (task is currently unsupported)

\medskip

\texttt{-eld}
    \tabto{2cm} use the Eldarica~\cite{hojjat2018eldarica} solver, packaged as version 2.0.4, invoked with \\
    \tabto{2cm} \texttt{java -Xmx14g -Xss20m -cp eld.jar lazabs.Main <file>}
    

    
\paragraph{Contributors \& Acknowledgement.}

\Korn is developed and maintained by the author.
The source code is available via \url{https://github.com/gernst/korn} under the MIT license.

Gregor Alexandru~\cite{alexandru2019} and Johannes Blau have contributed
key insights to the methodology outlined in \cref{sec:approach}.

\bibliographystyle{splncs04}
\bibliography{korn.bib}

\end{document}